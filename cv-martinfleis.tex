%!TeX program = pdflatex
% Martin Fleischmann's Curriculum Vitae
% Email: martin@martinfleischmann.net
% Web: https://martinfleischmann.net/
% Repo: https://github.com/martinfleis/cv

\documentclass[12pt,a4paper]{report}

\usepackage[T1]{fontenc} % output T1 font encoding (8-bit) for accented characters as single glyph
\usepackage[strict,autostyle]{csquotes} % smart and nestable quote marks
\usepackage[UKenglish]{babel} % regionalize hyphens, quote marks, etc automatically
\usepackage{microtype}% improve text appearance with kerning, etc
\usepackage{datetime} % enable formatting of date output
\usepackage{tabto}    % make nice tabbing
\usepackage{hyperref} % enable hyperlinks and pdf metadata
\usepackage{geometry} % manually set page margins
\usepackage{enumitem} % enumerate with [resume] option
\usepackage{titlesec} % allow custom section fonts
\usepackage{setspace} % custom line spacing

% what is your name?
\newcommand{\myname}{Martin Fleischmann}

% select default typefaces
\usepackage{ebgaramond} % document's serif typeface
\usepackage{helvet}     % document's sans serif typeface

% how far to tab for list items with left-aligned date: different fonts need different widths
\newcommand{\listtabwidth}{1.7cm}

% define font to use as document's title
\newcommand{\namefont}[1]{{\normalfont\bfseries\Huge{#1}}}

% set section heading fonts and before/after spacing
\SetTracking{encoding=*, family=\sfdefault}{30} % increase sans serif headings tracking
\titleformat{\section}{\lsstyle\sffamily\small\bfseries\uppercase}{}{}{}{}
\titlespacing{\section}{0pt}{30pt plus 4pt minus 4pt}{8pt plus 2pt minus 2pt}

% set subsection heading fonts and before/after spacing
\titleformat{\subsection}{\lsstyle\sffamily\footnotesize\bfseries}{}{}{}{}
\titlespacing{\subsection}{0pt}{16pt plus 4pt minus 4pt}{4pt plus 2pt minus 2pt}

% set page margins (assumes letter paper)
\geometry{body={6.5in, 9.0in},
    left=1.0in,
    top=1.0in}

% prevent paragraph indentation
\setlength\parindent{0em}

% set line spacing
\setstretch{0.9}

% define space between list items
\newcommand{\listitemspace}{0.25em}

% make unordered lists without bullets and use compact spacing
\renewenvironment{itemize}
{\begin{list}{}{\setlength{\leftmargin}{0em}
                \setlength{\parskip}{0em}
                \setlength{\itemsep}{\listitemspace}
                \setlength{\parsep}{\listitemspace}}}
{\end{list}}

% make tabbed lists so content is left-aligned next to years
\TabPositions{\listtabwidth}
\newlist{tablist}{description}{3}
\setlist[tablist]{leftmargin=\listtabwidth,
    labelindent=0em,
    topsep=0em,
    partopsep=0em,
    itemsep=\listitemspace,
    parsep=\listitemspace,
    font=\normalfont}

% print only the month and year when using \today
\newdateformat{monthyeardate}{\monthname[\THEMONTH] \THEYEAR}

% define hyperlink appearance and metadata for pdf properties
\hypersetup{
    colorlinks  = true,
    urlcolor    = black,
    pdfauthor   = {\myname},
    pdfkeywords = {urban morphology, urban geography, software development, geographic data science},
    pdftitle    = {\myname: Curriculum Vitae},
    pdfsubject  = {Curriculum Vitae},
    pdfpagemode = UseNone
}

\begin{document}
    \raggedright{}

    % display your name as the document title
    \namefont{\myname}

    % affiliation and contact info blocks
    \vspace{1em}

    \begin{minipage}[t]{0.6\textwidth}
        \begin{flushleft}
            \textsuperscript{1}Urban and Regional Laboratory, Charles University, CZ \\
            \textsuperscript{2}Geographic Data Science Lab, University of Liverpool, UK \\
            \textsuperscript{3}UrbanDataLab AG, Zürich, CH
        \end{flushleft}
    \end{minipage}%
    %
    \begin{minipage}[t]{0.4\textwidth}
        \begin{flushright}
            \href{mailto:martin@martinfleischmann.net}{martin@martinfleischmann.net} \\
            +420 774 627 733 \\
            \href{https://martinfleischmann.net}{martinfleischmann.net} \\
            \href{https://twitter.com/martinfleis}{@martinfleis} \\
            \href{https://orcid.org/0000-0003-3319-3366}{0000-0003-3319-3366}
        \end{flushright}
    \end{minipage}%


    \section*{Education}

    \begin{tablist}

        \item[Ph.D.] \tab{}Architecture, University of Strathclyde, Glasgow, 2021 \\
                           \textit{The Urban Atlas: Methodological Foundation of a Morphometric Taxonomy of Urban Form}
        \item[MSc.]  \tab{}Urban Design, University of Strathclyde, Glasgow, 2017 \\
                           \textit{Measuring Urban Form: A Systematisation of Attributes for Quantitative Urban Morphology}
        \item[BSc.]  \tab{}Architecture and Urbanism, Czech Technical University, Prague, 2015

    \end{tablist}


    \section*{Professional Appointments}

    \begin{tablist}

        \item[2022--] \tab{}UrbanDataLab AG \\
                            Head of Data and Analytics \\

    \end{tablist}


    \section*{Academic Appointments}

    \begin{tablist}

         \item[2022--] \tab{}Charles University in Prague \\
                            Postdoctoral Researcher, Department of Social Geography and Regional Development \\
                            Urban and Regional Laboratory


        \item[2020--22] \tab{}University of Liverpool \\
                            Research Associate, Department of Geography and Planning \\
                            \textit{"Learning an urban grammar from satellite data through AI."} \\
                            Economic and Social Research Council \& Alan Turing Institute

    \end{tablist}


    \section*{Honorary Appointments}

    \begin{tablist}

        \item[2022--] \tab{}University of Liverpool \\
                            Fellow, Department of Geography and Planning \\

    \end{tablist}

    \section*{Publications}

    \subsection*{Journal Articles}

    \begin{tablist}

        \item[2022] \tab{}Rowe, F., Calafiore, A., Arribas-Bel, D., Samardzhiev, K., and Fleischmann, M. \enquote{Urban exodus? Understanding human mobility in Britain during the COVID-19 pandemic using Meta-Facebook data} \textit{Population, Space and Place}, e2637. \href{https://doi.org/10.1002/psp.2637}{doi:10.1002/psp.2637}

        \item[2022] \tab{}Arribas-Bel, D. and Fleischmann, M. \enquote{Spatial Signatures: Understanding (urban) spaces through form and function} \textit{Habitat International}, 128, (102641). \href{https://doi.org/10.1016/j.habitatint.2022.102641}{doi:10.1016/j.habitatint.2022.102641}

        \item[2022] \tab{}Fleischmann, M. and Arribas-Bel, D. \enquote{Geographical characterisation of British urban form and function using the spatial signatures framework} \textit{Scientific Data}, 9, (546). \href{https://doi.org/10.1038/s41597-022-01640-8}{doi:10.1038/s41597-022-01640-8}

        \item[2022] \tab{}Venerandi, A., Feliciotti, A. ,Fleischmann, M., Kourtit, K., Porta, S.\enquote{Urban form character and Airbnb in Amsterdam \(NL\): A morphometric approach} \textit{Environment and Planning B: Urban Analytics and City Science}. \href{https://doi.org/10.1177/23998083221115196}{doi:10.1177/23998083221115196}

        \item[2022] \tab{}Samardzhiev, K., Fleischmann, M., Arribas-Bel, D., Calafiore, A., Rowe, F. \enquote{Functional Signatures in Great Britain: A dataset.} \textit{Data in Brief}, 108335, \href{https://doi.org/10.1016/j.dib.2022.108335}{doi:10.1016/j.dib.2022.108335}

        \item[2022] \tab{}Singleton, A., Arribas-Bel, D., Murray, J., and Fleischmann, M. \enquote{Estimating generalized measures of local neighbourhood context from multispectral satellite images using a convolutional neural network.} \textit{Computers, Environment and Urban Systems}, 95, 101802. \href{https://doi.org/10.1016/j.compenvurbsys.2022.101802}{doi:10.1016/j.compenvurbsys.2022.101802}

        \item[2021] \tab{}Fleischmann, M., Feliciotti, A., Romice, O. and Porta, S. \enquote{Methodological Foundation of a Numerical Taxonomy of Urban Form.} \textit{Environment and Planning B: Urban Analytics and City Science} 49 (4), 1283-1299. \href{https://doi.org/10.1177/23998083211059835}{doi:10.1177/23998083211059835}

        \item[2021] \tab{}Fleischmann, M., Feliciotti, A. and Kerr, W. \enquote{Evolution of urban patterns: urban morphology as an open reproducible data science.} \textit{Geographical Analysis} 54 (3). \href{https://doi.org/10.1111/gean.12302}{doi:10.1111/gean.12302}

        \item[2020] \tab{}Dal Cin, F., Fleischmann, M., Romice, O. and Costa, J.P. \enquote{Climate Adaptation Plans in the Context of Coastal Settlements: The Case of Portugal.} \textit{Sustainability} 12 (20). \href{https://doi.org/10.3390/su12208559}{doi:10.3390/su12208559}

        \item[2020] \tab{}Fleischmann, M., Romice, O. and Porta, S. \enquote{Measuring urban form: overcoming terminological inconsistencies for a quantitative and comprehensive morphologic analysis of cities.} \textit{Environment and Planning B: Urban Analytics and City Science} 48 (8), 2133-2150. \href{https://doi.org/10.1177/2399808320910444}{doi:10.1177/2399808320910444}

        \item[2020] \tab{}Fleischmann, M., Feliciotti, A., Romice, O. and Porta, S. \enquote{Morphological tessellation as a way of partitioning space: Improving consistency in urban morphology at the plot scale} \textit{Computers, Environment and Urban Systems} 80, 101441. \href{https://doi.org/10.1016/j.compenvurbsys.2019.101441}{doi:10.1016/j.compenvurbsys.2019.101441}

        \item[2019] \tab{}Fleischmann, M. \enquote{momepy: Urban Morphology Measuring Toolkit} \textit{Journal of Open Source Software} 4 (43), 1807. \href{https://doi.org/10.21105/joss.01807}{doi:10.21105/joss.01807}


    \end{tablist}

    \subsection*{Conference Papers}

    \begin{tablist}

        \item[2022] \tab{}Fleischmann, M. and Arribas-Bel, D. \enquote{Classifying urban form at national scale : the British morphosignatures}, \textit{Annual Conference Proceedings of the XXVIII International Seminar on Urban Form}. University of Strathclyde Publishing, Glasgow, pp. 895-905. ISBN 9781914241161. \href{https://doi.org/10.17868/strath.00080527}{doi:10.17868/strath.00080527}

        \item[2022] \tab{}Wang, J., Fleischmann, M., Venerandi, A., Kuffer, M., Porta, S. \enquote{Earth observation+ morphometrics: towards a systematic understanding of cities in challenging contexts}, \textit{Annual Conference Proceedings of the XXVIII International Seminar on Urban Form}. University of Strathclyde Publishing, Glasgow, pp. 363-370. ISBN 9781914241161. \href{https://doi.org/10.17868/strath.00080476}{doi:10.17868/strath.00080467}

        \item[2022] \tab{}Fleischmann, M., Romice, O. and Porta, S. \enquote{Applicability of morphological tessellation and its topological derivatives in the quantitative analysis of urban form} \textit{Cities as Assemblages, Proceedings of XXVI International Seminar on Urban Form, volume 3}. Nicosia. \href{https://doi.org/10.36158/978889295357413}{doi:10.36158/978889295357413}

    \end{tablist}

    \subsection*{Reports and Other publications}

    \begin{tablist}

        \item[2022] \tab{}Feliciotti, A., Fleischmann, M., eds. \enquote{ISUF Annual Conference Proceedings of the XXVII International Seminar on Urban Form:" Urban Form and the Sustainable and Prosperous City"}. University of Strathclyde Publishing, Glasgow, ISBN 9781914241161. \href{https://doi.org/10.17868/80146}{doi:10.17868/80146}
        \item[2022] \tab{}Rowe, F., Arribas-Bel, D., Calafiore, A., MacDonald, J., Samardzhiev, K., Fleischmann, M., \enquote{Mobility data in urban science. Workshop report}. The Alan Turing Institute. London, United Kingdom.
        \item[2021] \tab{}Fleischmann, M. \enquote{Policy Brief: Flexibilita prostorového uspořádání města} (Flexibility of the spatial configuration of the city) \textit{Územně analytické podklady hl. m. Prahy}. Institute for Planning and Development, Prague, Czechia
        \item[2021] \tab{}Darlington-Pollock, F., Arribas-Bel, D., Fleischman, M., Newsham, N., Rowe, F. \enquote{Policy Brief: What do ‘left behind’ areas look like over time? Developing place-based typologies of left behind areas}. Local Trust, UK
        \item[2020] \tab{}Places Platform, Smart Growth America, UDSU. \enquote{Welcome to the Future of Grand Rapids: Assessing Place-Based Economic, Social Equity, and Public Health Performance}. Downtown Grand Rapids Inc., US
        \item[2018] \tab{}Fleischmann, M. \enquote{Measuring Urban Form} \textit{URBAN DESIGN} 146 (Spring 2018), 6-7.

    \end{tablist}

    \section*{Research software development}

    \begin{tablist}

        \item[2021--] \tab{}Xvec: vector data cubes for Xarray. (author and maintainer)
        \item[2021--] \tab{}xyzservices: source of XYZ tiles providers. (author and maintainer)
        \item[2020--] \tab{}clustergram: visualization and diagnostics for cluster analysis (author and maintainer)
        \item[2020--] \tab{}PySAL: Python Spatial Analysis Library (core developer)
        \item[2019--] \tab{}GeoPandas: an open source project to make working with geospatial data in python easier (core developer)
        \item[2018--] \tab{}momepy: Urban Morphology Measuring Toolkit (author and maintainer)

    \end{tablist}

    \section*{Editorial Appointments}

    \begin{tablist}

        \item[2021--] \tab{}Journal of Open Source Software \\
                            Editor

    \end{tablist}

    \section*{Research income}

    \subsection*{Awards}

    \begin{tablist}

        \item[2020] \tab{}\textit{NumFOCUS} \enquote{Improvement and modernization of GeoPandas documentation}, \$5,000
        \item[2018] \tab{}\textit{University of Strathclyde} \enquote{John Anderson Research Award}, ~\£60,000

    \end{tablist}

    \section*{Invited Talks}

    \begin{tablist}

        \item[2021] \tab{}\enquote{Capturing the Structure of Cities with Data Science.} Spatial Data Science Conference 2021. Online. Oct 26.

        \item[2021] \tab{}\enquote{Spatial Signatures: Dynamic classification of the built environment.} Spatial Analytics and Data seminar series by University of Newcastle and University of Bristol. Online. Mar 30.

        \item[2020] \tab{}\enquote{Reading cities as numbers. Where data science meets urbanism.} Academy of Urbanism (Scotland). Online. Dec 3.

    \end{tablist}


    \section*{Conference Activity}


    \subsection*{Conferences Organized}

    \begin{tablist}

        \item[2021] \tab{}\enquote{XXVIII International Seminar on Urban Form} Urban Form and the Sustainable Prosperous City. Glasgow (virtually), UK. Jun 29 -- Jul 3.

    \end{tablist}

    \subsection*{Conference Papers Presented}

    Presenting author \textit{italicized} if other than first author.\bigskip

    \begin{tablist}

        \item[2021] \tab{}Fleischmann, M. and Arribas-Bel, D. \enquote{Classifying urban form at a national scale: The case of Great Britain} XXVIII International Seminar on Urban Form: Urban Form and the Sustainable Prosperous City. Glasgow/Virtual. \@ Jun 29 -- Jul 3.

        \item[2021] \tab{}Venerandi, A., \textit{Feliciotti, A.}, Fleischmann, M., Kourtit, K., Romice, O., Nijkamp, P., Porta, S. and Fusco, G. \enquote{Urban form and Airbnb: a study of their spatial relation in Amsterdam (NL)} XXVIII International Seminar on Urban Form: Urban Form and the Sustainable Prosperous City. Glasgow/Virtual. \@ Jun 29 -- Jul 3.

        \item[2021] \tab{}Porta, S., \textit{Venerandi, A.}, Feliciotti, A., Fleischmann, M. and Romice, O. \enquote{A Numerical Taxonomy of Urban Form in London: an Urban Morphometric Approach} XXVIII International Seminar on Urban Form: Urban Form and the Sustainable Prosperous City. Glasgow/Virtual. \@ Jun 29 -- Jul 3.

        \item[2021] \tab{}Wang, J., Fleischmann, M., Venerandi, A., Kuffer, M. and Porta, S. \enquote{Earth Observation + Morphometrics: towards a systematic understanding of cities in challenging contexts} XXVIII International Seminar on Urban Form: Urban Form and the Sustainable Prosperous City. Glasgow/Virtual. \@ Jun 29 -- Jul 3.

        \item[2020] \tab{}Fleischmann, M. \enquote{On the morphological composition of cities and how to measure it.} GeoPython. Virtual. \@ Sep 21--22.

        \item[2020] \tab{}Fleischmann, M., Romice, O. and Porta, S. \enquote{The Urban Atlas: Towards a Morphometric Taxonomy of Urban Form} XXVII International Seminar on Urban Form: Cities in the 21st Century. Salt Lake City/Virtual. \@ Sep 1--4.

        \item[2020] \tab{}Hořák, D., Fleischmann, M., Romice, O. and Porta, S. \enquote{Just like a bird. A community ecology perspective on urban form evolution} XXVII International Seminar on Urban Form: Cities in the 21st Century. Salt Lake City/Virtual. \@ Sep 1--4.

        \item[2020] \tab{}Feliciotti, A., Fleischmann, M., Romice, O., Kerr, W. and Porta, S. \enquote{Morphological Resilience Evaluation (MoRE): a new assessment framework for multi-level assessment of urban form resilience} XXVII International Seminar on Urban Form: Cities in the 21st Century. Salt Lake City/Virtual. \@ Sep 1--4.

        \item[2020] \tab{}Dal Cin, F., Fleischmann, M., Romice, O. and Costa, J.P. \enquote{DECODING SEASHORE STREETS: urban morphometrics as a tool for adaptation measures} XXVII International Seminar on Urban Form: Cities in the 21st Century. Salt Lake City/Virtual. \@ Sep 1--4.

        \item[2020] \tab{}Fleischmann, M., Dal Cin, F., Romice, O. and Barreiros Proença, S. \enquote{Understanding seashore streets: urban morphometrics as a tool for climate-induced risk assessment} ISUF Italy International Conference: Urban Substrata and City Regeneration. Rome \@ Feb 19--22.

        \item[2019] \tab{}Feliciotti, A., Fleischmann, M., Romice, O., and Porta, S.  \enquote{Morphological Resilience: from a theory of resilient urban forms to the tools for its implementation} 12th CITTA International Conference on Planning Research: Spatial Planning for Change. Porto. \@ Sep 19--20.

        \item[2019] \tab{}Fleischmann, M., Romice, O. and Porta, S. \enquote{Applicability of morphological tessellation and its topological derivatives in the quantitative analysis of urban form} XXVI International Seminar on Urban Form: Cities as Assemblages. Nicosia. \@ Jul 2--6.

        \item[2019] \tab{}Fleischmann, M., Hořák, D., Porta, S. et al. \enquote{Ecological and evolutionary perspectives in urban morphology} Ecology and Evolution of Urban Areas, CEE Symposium. London. \@ Jun 13.


    \end{tablist}

    \subsection*{Other conference activities}

    \begin{tablist}

        \item[2022] \tab{}Fleischmann, M. and van den Bossche, J. \enquote{Scaling up vector analysis with Dask-GeoPandas} GeoPython 2022, Basel, CH, Jun 20 -- 22.
        \item[2022] \tab{}Nieves, J., Fleischmann, M. and Calafiore, A. \enquote{Dasymetric population modelling in R and Python workshop} GISRUK 2022, Liverpool UK. Apr 6 -- 8.
        \item[2021] \tab{}Fleischmann, M. and Arribas-Bel, D. \enquote{Spatial Signatures in Great Britain - lightning talk} Towards urban analytics 2.0, Alan Turing Institute (Urban Analytics) event, Leeds UK. Nov 30 -- Dec 1.

    \end{tablist}


    \section*{Service}

    \subsection*{Academic Journal Peer Review}

    \begin{itemize}

        \item \textit{Cartography and Geographic Information Science}
        \item \textit{Environment and Planning B:\ Urban Analytics and City Science}
        \item \textit{Geographical Analysis}
        \item \textit{Journal of Geographical Systems}
        \item \textit{Journal of Open Research Software}
        \item \textit{Journal of Open Source Software}
        \item \textit{Journal of Statistical Software}
        \item \textit{Moravian Geographical Reports}
        \item \textit{Nature Machine Intelligence}
        \item \textit{Plos ONE}
        \item \textit{Urban Morphology}

    \end{itemize}

    \subsection*{Academic Press Peer Review}

    \begin{itemize}

        \item \textit{CRC Press}

    \end{itemize}

    \subsection*{Other Peer Review}

    \begin{itemize}

        \item \textit{pyOpenSci}
        \item \textit{FA CTU Doctoral proposals}

    \end{itemize}

    \subsection*{Other}

    \begin{tablist}

        \item[2022] \tab{}Lecturer, OpenGeoHub Summer School Siegburg \\
                          \textit{Introduction to GeoPandas and its Python ecosystem}

        \item[2022] \tab{}Mentor, Google Summer of Code. \\
                          \textit{PySAL - Street network simplification projects} \\
                          Greg Maya, Gabriel Agostini

        \item[2021] \tab{}Mentor, Google Summer of Code. \\
                          \textit{Geopandas - Dask bridge to scale geospatial analysis} \\
                          Thomas Statham (University of Bristol)

        \item[2019, 2022] \tab{}External reviewer, FA CTU Doctoral workshop

    \end{tablist}

    \section*{Consulting}

    \begin{tablist}

        \item[2022] \tab{}UrbanDataLab AG, Switzerland
        \item[2021] \tab{}All-Party Parliamentary Group for 'left behind' neighbourhoods, UK
        \item[2019--20] \tab{}Institute for Planning and Development, Prague, Czechia
        \item[2020] \tab{}Places Platform, US

    \end{tablist}


    \section*{Memberships}

    \begin{tablist}

        \item[2021--] \tab{}Royal Geographical Society (Fellow)
        \item[2019--] \tab{}International Seminar on Urban Form (Member)
        \item[2018--] \tab{}Academy of Urbanism (Young Urbanist)

    \end{tablist}

    % display today's date as Month Year after a vertical space below the end of the text
    \begin{center}
        \vfill
        Updated \monthyeardate\today
    \end{center}

\end{document}
